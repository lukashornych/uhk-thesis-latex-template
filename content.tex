\section{Úvod}

% TODO WRITER: text

Zde vysvětlit problémovou situaci a otázky, které se budou v bakalářské/diplomové práci řešit.

\section{Cíl práce}

% TODO WRITER: text

Smysl a účel, výzkumné otázky.

\section{Metodika zpracování}

% TODO WRITER: text

Cíle, hypotézy/ výzkumné otázky, způsob hledání odpovědí na výzkumné otázky včetně metodiky vlastního výzkumu/šetření, literární rešerše.

\section{Vlastní text práce}

% TODO WRITER: text

Vlastní řešení dokládá student zpravidla v několika kapitolách. Podle charakteru práce musí student uvážit, zda informace
netextové povahy (data, tabulky, obrázky atd.) bude uvádět přímo v textu, nebo je zařadí až za celou práci ve formě příloh, či bude kombinovat oba způsoby.
Více podrobností viz Metodické pokyny pro vypracování bakalářských a diplomových prací (zveřejňované formou výnosů děkana)
a v kurzu MES – Metodologický seminář.

	\subsection{Podkapitola}

	Text podkapitoly.

	Následuje ukázka nějakého seznamu:
	\begin{itemize}
		\item fotografie/avatar,
		\item kategorizační štítky,
		\item popisek.
	\end{itemize}

		\subsubsection{Podpodkapitola}

		Lorem ipsum dolor sit amet, consectetur adipiscing elit. Phasellus sit amet ornare diam, id consequat diam.

			\nlparagraph{Paragraf}

			\noindent Ukázka prvního odstavce v paragrafu. Lorem ipsum dolor sit amet, consectetur adipiscing elit. Phasellus sit amet ornare diam, id consequat diam.

			Následuje použití citací: Citace \cite{html_hypertext_markup_language}, \cite{hibernate_docs}, \cite{ddd_quickly}.
			Použití zkratek se rozděluje na první použítí a další použití: první použití \firstac{URL} a další použítí \ac{URL}.

			Použití obrázku je jednoduché, zadáme název souboru, šířku, popisek a zdroj:

			\cntcapfigure{rovnovaha_paka}{8cm}{Páka rovnováhy vzhledu.}{\cite{vizualni_rovnovaha}}

			V případně, že autor je i autor obrázku:

			\cntcapfigure{uhk}{\linewidth}{Toto je UHK.}{[autor]}

			Blok kódu může vypadat takto:

			\begin{codeblock}
				\begin{verbatim}
@Mapper
public interface AccountDao {
  @Select({
    "select *",
    "from " + Account.TABLE_NAME,
    "where email = #{email}"
  })
  Optional<Account> findByEmail(String email);
}
				\end{verbatim}
				\captionsource{Ukázka bloku kódu.}{[autor]}
			\end{codeblock}

			Tabulka zas může vypadat takto:

			\begin{table}[hbt!]
				\captionsource{Ukázková tabulka.}{[autor]}
				\centering
				\begin{tabular}{| l | r | r | r | }
					\hline
					&        psnr &      ssim &      doba  \\
					model &       (db)    &           & gen. (s) \\
					\hline
					bik. int. & 28.3155 & 0.8566 & 0.0322 \\
					nn1000    & 30.1461 & 0.9043 & 0.8109 \\
					nn1001    & 30.0324 & 0.9023 & 0.7486 \\
					nn1002    & \textbf{30.1886} & \textbf{0.9046} & 1.1731 \\
					nn1003    & 30.0390 & 0.9030 & 1.1320 \\
					nn1004    & 24.9772 & 0.7172 & 4.4367 \\
					nn1005    & 26.1629 & 0.8004 & 4.0475 \\
					nn1006    & 27.9129 & 0.8438 & 4.0683 \\
					nn1007    & 27.5834 & 0.8360 & 4.2082 \\
					\hline
				\end{tabular}
			\end{table}

			\newpage

			Další text

\section{Závěry a doporučení}

% TODO WRITER: text

Kritická diskuze nad výsledky, ke kterým autor dospěl (soulad výsledků  literaturou či předpoklady;
výsledky a okolnosti, které zvláště ovlivnily předkládanou práci atd.). Je vhodné naznačit i případné další
(popř. alternativní) možnosti zkoumání dané problematiky a otevřené problémy pro další studium.
